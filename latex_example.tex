\documentclass[12pt, letterpaper]{article}
\usepackage[left=1.5cm, right=1.5cm]{geometry}

\usepackage{amsmath}
\usepackage{amssymb}
\usepackage{mathbbol}
\usepackage{graphicx}
\usepackage{hyperref}
\usepackage{fancyhdr}

\DeclareMathOperator{\sech}{sech}
\DeclareMathOperator{\csch}{csch}
\DeclareMathOperator{\arcsec}{arcsec}
\DeclareMathOperator{\arccot}{arccot}
\DeclareMathOperator{\arccsc}{arccsc}
\DeclareMathOperator{\arccosh}{arccosh}
\DeclareMathOperator{\arcsinh}{arcsinh}
\DeclareMathOperator{\arctanh}{arctanh}
\DeclareMathOperator{\arcsech}{arcsech}
\DeclareMathOperator{\arccsch}{arccsch}
\DeclareMathOperator{\arccoth}{arccoth}
\DeclareMathOperator{\projhelper}{proj}
\DeclareMathOperator*{\R}{\mathbb{R}}

\newcommand{\pderiv}[2]{\frac{\partial #1}{\partial #2}}

\title{Topics in Math: Writeup 9}
\author{Avi Gupta}
\date{}
\begin{document}

\pagestyle{fancy}
\fancyhf{}
\fancyhead[L]{\textbf{Avi Gupta}}
\fancyhead[R]{\textbf{savim2020@gmail.com}}
\fancyfoot[R]{Page {\thepage}}

This {\LaTeX} example is from a mathematics writeup I did in high school,
for a Calculus III course. If you have any questions, feel free
to contact me at {\href{mailto:savim2020@gmail.com}{savim2020@gmail.com}}.

% \maketitle
\begin{enumerate}
    \item Let $G$ be the rectangular box defined by the inequalities
    $a \le x \le b$,
    $c \le y \le d$,
    $k \le z \le l$.
    Show that
    \[\iiint\limits_G f(x)g(y)h(z)\ dV =
    \left[\int_a^b f(x)\ dx\right]
    \left[\int_c^d g(y)\ dy\right]
    \left[\int_k^l h(z)\ dz\right]
    \]
    \textbf{Answer}

    We will define some values:
    \[X = \int_a^b f(x)\ dx\]
    \[Y = \int_c^d g(y)\ dy\]
    \[Z = \int_k^l h(z)\ dz\]
    Because $X$, $Y$, and $Z$ are all definite integrals with constant bounds,
    they are all constants.
    \[
    \left[\int_a^b f(x)\ dx\right]
    \left[\int_c^d g(y)\ dy\right]
    \left[\int_k^l h(z)\ dz\right]= XYZ\]
    \[\iiint\limits_G f(x)g(y)h(z)\ dV = \int_k^l \int_c^d \int_a^b f(x)g(y)h(z)\ dx\ dy\ dz\]
    $g(y)$ and $h(z)$ are treated as constants with respect to $x$. We can then rewrite this as
    \[\int_k^l \int_c^d \left[g(y)h(z) \int_a^b f(x)\ dx\right]\ dy\ dz\]
    \[ = \int_k^l \int_c^d \left(g(y)h(z) X\right)\ dy\ dz\]
    \[ = X \int_k^l \int_c^d g(y)h(z)\ dy\ dz\]
    $h(z)$ is treated as a constant with respect to $y$. We can then rewrite this as
    \[X \int_k^l \left[h(z) \int_{c}^{d} g(y)\ dy\right]\ dz\]
    \[= X \int_k^l  h(z) Y\ dz\]
    \[= XY \int_k^l  h(z) \ dz\]
    \[= XYZ \blacksquare\]

    \item Use the result of the previous exercise to evaluate the following integrals.
      \begin{enumerate}
        \item $\displaystyle \iiint\limits_G xy^2 \sin z\ dV$
        where $\displaystyle G = \left\{(x, y, z) |
        -1 \le x \le 1,
        0 \le y \le 1,
        0 \le z \le \frac{\pi}{2}\right\}$.

        \textbf{Answer}

        \[\iiint\limits_G xy^2 \sin z\ dV
        = \int_{-1}^{1} \int_{0}^{1} \int_{0}^{\frac{\pi}{2}} xy^2 \sin z\ dz\ dy\ dx\]
        \[ =
        \left[\int_{-1}^{1} x\ dx\right]
        \left[\int_{0}^{1} y^2\ dy\right]
        \left[\int_{0}^{\frac{\pi}{2}} \sin z\ dz\right]\]
        Because $x$ is an odd function, $\displaystyle \int_{-1}^{1} x\ dx = 0$.
        And because $y^2$ and $\sin z$ are continuous fuctions along all real numbers,
        their integrals exist and are defined.
        \[ = 0 \left[\int_{0}^{1} y^2\ dy\right]
        \left[\int_{0}^{\frac{\pi}{2}} \sin z\ dz\right]\]
        \[ = 0\]

        \item $\displaystyle \iiint\limits_G e^{2x + y - z}\ dV$
        where $\displaystyle G = \left\{(x, y, z) |
        0 \le x \le 1,
        0 \le y \le \ln 3,
        0 \le z \le \ln 2 \right\}$.

        \textbf{Answer}

        \[\iiint\limits_G e^{2x + y - z}\ dV = \iiint\limits_G
        \left(e^{2x}\right)
        \left(e^{y}\right)
        \left(e^{-z}\right)
        \ dV
        \]
        \[ = \int_{0}^{1} \int_{0}^{\ln 3} \int_{0}^{\ln 2}
        \left(e^{2x}\right)
        \left(e^{y}\right)
        \left(e^{-z}\right)
        \ dz\ dy\ dx
        \]
        \[=
        \int_{0}^{1} e^{2x}\ dx
        \int_{0}^{\ln 3} e^{y}\ dy
        \int_{0}^{\ln 2} e^{-z}\ dz\]

        \[=
        \left[\frac{1}{2} e^{2x}\right]_{x=0}^1
        \left[e^y\right]_{y=0}^{\ln 3}
        \left[-e^{-z}\right]_{z=0}^{\ln 2}\]

        \[ = \frac{1}{2}(e^2 - 1) \left(e^{\ln 3} - 1\right) \left(-e^{-\ln 2} + 1\right)\]
        \[ - \frac{1}{2}(e^2 - 1) (3 - 1)\left(-\frac{1}{2} + 1\right)\]
        \[ = \frac{1}{2}(e^2 - 1)(2)\left(\frac{1}{2}\right)\]
        \[ = \frac{e^2 - 1}{2}\]
      \end{enumerate}

      \item Let $G$ be the solid in the first octant bounded by the sphere
      $x^2 + y^2 + z^2 = 4$
      and the coordinate planes. Find
      \[\iiint\limits_G xyz\ dV\]
      using
      \begin{enumerate}
        \item Rectangular coordinates

        \textbf{Answer}

        Because we are in the first octant, $x \ge 0$, $y \ge 0$, and $z \ge 0$.
        \[G = \left\{ (x, y, z) | 0 \le x \le 2,
        0 \le y \le \sqrt{4 - x^2},
        0 \le z \le \sqrt{4 - x^2 - y^2} \right\}\]
        \[\iiint\limits_G xyz\ dV =
        \int_{0}^{2} \int_{0}^{\sqrt{4 - x^2}} \int_{0}^{\sqrt{4 - x^2 - y^2}} xyz\ dz\ dy\ dx\]
        \[ = \int_{0}^{2} \int_{0}^{\sqrt{4 - x^2}}
        \left[\frac{1}{2} xyz^2\right]_{z = 0}^{\sqrt{4 - x^2 - y^2}}\ dy\ dx\]
        \[ = \frac{1}{2} \int_{0}^{2} \int_{0}^{\sqrt{4 - x^2}} xy(4 - x^2 - y^2)\ dy\ dx\]
        \[ = \frac{1}{2} \int_{0}^{2} \int_{0}^{\sqrt{4 - x^2}} (4xy - x^3 y - x y^3 )\ dy\ dx\]
        \[ = \frac{1}{2} \int_{0}^{2}
        \left[2xy^2 - \frac{1}{2}x^3y^2 - \frac{1}{4}xy^4\right]_{y = 0}^{\sqrt{4 - x^2}}\ dx\]
        \[ = \frac{1}{2} \int_{0}^{2}
        \left(2x(4 - x^2) - \frac{1}{2}x^3(4-x^2) - \frac{1}{4}x(4 - x^2)^2\right)\ dx\]
        \[ = \frac{1}{2} \int_{0}^{2}
        \left((8x - 2x^3) - \frac{1}{2}(4x^3 - x^5) - \frac{1}{4}(16x - 8x^3 + x^5)\right)\ dx
        \]
        \[ = \frac{1}{2} \int_{0}^{2}
        \left( 8x - 2x^3 - 2x^3 + \frac{1}{2}x^5 - 4x + 2x^3 - \frac{1}{4}x^5\right)\ dx
        \]
        \[ = \frac{1}{2} \int_{0}^{2}\left(\frac{1}{4}x^5 - 2x^3 + 4x\right)\ dx\]
        \[ = \frac{1}{2} \left[\frac{1}{24}x^6 - \frac{1}{2}x^4 + 2x^2\right]_0^2\]
        \[ = \frac{1}{2} \left(\frac{64}{24} - \frac{16}{2} + 2(2)^2\right)\]
        \[ = \frac{1}{2} \left(\frac{8}{3} - 8 + 8\right)\]
        \[ = \frac{1}{2} \cdot \frac{8}{3} = \frac{4}{3}\]
        % \[ = \frac{1}{2} \int_{0}^{2}
        % \left((8x - 4x^2) - \frac{1}{2}(4x^3-x^5) - \frac{1}{4}(16 - 8x^2 + x^4)\right)\ dx\]
        % \[ = \frac{1}{2} \int_{0}^{2}
        % \left( \frac{1}{2}x^5 - \frac{1}{4}x^4 - 2x^3 - 2x^2 + 8x - 4\right)\ dx\]
        % \[ = \frac{1}{2} \left[
        %   \frac{1}{12}x^6 - \frac{1}{20}x^5 - \frac{1}{2}x^4 - \frac{2}{3}x^3 + 4x^2 - 4x
        % \right]_0^2\]
        % \[ = \frac{1}{2} \left(
        %   \frac{64}{12} - \frac{32}{20} - \frac{16}{2} - \frac{16}{3} + 4(4) - 4(2)
        % \right)\]
        % \[ = \frac{32}{12} - \frac{16}{20} - \frac{8}{2} - \frac{8}{3} + 2(4) - 2(2)\]
        % \[ = \frac{8}{3} - \frac{4}{5} - 4 - \frac{8}{3} + 8 - 4\]

        \item Cylindrical Cooridnates

        \textbf{Answer}

        The sphere has a radius of $2$. Because we are in the first octant,
        $\displaystyle 0 \le \theta \le \frac{\pi}{2}$.
        \[G = \left\{ (r, \theta, z) | 0 \le r \le 2,
        0 \le \theta \le \frac{\pi}{2},
        0 \le z \le \sqrt{4 - r^2} \right\}\]
        \[\iiint\limits_G xyz\ dV
        = \int_{0}^{\frac{\pi}{2}} \int_{0}^{2} \int_{0}^{\sqrt{4 - r^2}}
        \left(r \cos \theta\right) \left(r \sin \theta\right) zr\ dz\ dr\ d\theta\]
        \[ = \int_{0}^{\frac{\pi}{2}} \int_{0}^{2} \int_{0}^{\sqrt{4 - r^2}}
        r^3z \sin (\theta) \cos (\theta)\ dz\ dr\ d\theta\]
        \[ = \int_{0}^{\frac{\pi}{2}} \int_{0}^{2} \int_{0}^{\sqrt{4 - r^2}}
        \frac{1}{2}r^3z \sin (2\theta)\ dz\ dr\ d\theta\]
        \[ = \frac{1}{2}\int_{0}^{\frac{\pi}{2}} \int_{0}^{2}
        \left[\frac{1}{2}r^3 z^2 \sin (2\theta)\right]_{z = 0}^{\sqrt{4 - r^2}}\ dr\ d\theta\]
        \[ = \frac{1}{4} \int_{0}^{\frac{\pi}{2}} \int_{0}^{2}
        \left[r^3 z^2 \sin (2\theta)\right]_{z = 0}^{\sqrt{4 - r^2}}\ dr\ d\theta\]

        \[ = \frac{1}{4} \int_{0}^{\frac{\pi}{2}} \int_{0}^{2}
        r^3 (4 - r^2) \sin(2 \theta)\ dr\ d\theta\]
        \[ = \frac{1}{4}
        \left[\int_{0}^{\frac{\pi}{2}} \sin (2 \theta)\ d\theta\right]
        \left[\int_{0}^{2} (4r^3 - r^5)\ dr\right]
        \]
        \[ = \frac{1}{4}
        \left[- \frac{1}{2} \cos (2 \theta)\right]_0^{\frac{\pi}{2}}
        \left[r^4 - \frac{1}{6}r^6\right]_0^2\]
        \[ = -\frac{1}{8}
        \left[\cos (2 \theta)\right]_0^{\frac{\pi}{2}}
        \left(16 - \frac{64}{6}\right)\]
        \[ - \frac{1}{8} \left(-1 - 1\right) \left(16 - \frac{64}{6}\right)\]
        \[ = \frac{1}{4} \left(16 - \frac{64}{6}\right)\]
        \[ = 4 - \frac{16}{6}\]
        \[ = \frac{12}{3} - \frac{8}{3} = \frac{4}{3}\]

        \item Spherical coordinates

        \textbf{Answer}

        Because we are in the first octant, $\displaystyle 0 \le \theta \le \frac{\pi}{2}$
        and $\displaystyle 0 \le \varphi \le \frac{\pi}{2}$.
        \[G = \left\{ (\rho, \theta, \varphi) |
        0 \le \rho \le 2,
        0 \le \theta \le \frac{\pi}{2},
        0 \le \varphi \le \frac{\pi}{2} \right\}\]
        \[\iiint\limits_G xyz\ dV =
        \int_{0}^{\frac{\pi}{2}} \int_{0}^{\frac{\pi}{2}} \int_{0}^{2}
        (\rho \sin \varphi \cos \theta) (\rho \sin \varphi \sin \theta)(\rho \cos \varphi)
        (\rho^2 \sin \varphi)\ d\rho\ d\varphi\ d\theta\]
        \[ = \int_{0}^{\frac{\pi}{2}} \int_{0}^{\frac{\pi}{2}} \int_{0}^{2}
        \rho^5 \sin^3 \varphi \cos \varphi \sin \theta \cos \theta\ d\rho\ d\varphi\ d\theta
        \]
        \[ = \left[\int_{0}^{\frac{\pi}{2}} \sin \theta \cos \theta\ d\theta\right]
        \left[\int_{0}^{\frac{\pi}{2}} \sin^3 \varphi \cos \varphi\ d\varphi\right]
        \left[\int_{0}^{2} \rho^5\ d\rho\right]\]
        \[ = \left[\frac{1}{2}\sin^2 \theta\right]_0^{\frac{\pi}{2}}
        \left[\frac{1}{4} \sin^4 \varphi\right]_0^{\frac{\pi}{2}}
        \left[\frac{1}{6}\rho^6\right]_0^2\]
        \[ = \frac{1}{2} \cdot \frac{1}{4} \cdot \frac{64}{6}\]
        \[ = \frac{8}{6} = \frac{4}{3}\]

      \end{enumerate}

      \item Evaluate $\iint_R \sin(9x^2 + 4y^2)\ dA$, where $R$ is the region in the first quadrant
      bounded by the ellipse $9x^2 + 4y^2 = 1$, by making an appropriate change in variables.

      \textbf{Answer}

      Change $x$ and $y$ into terms of $u$ and $v$:
      \[x = \frac{u}{3}, \quad y = \frac{v}{2}\]
      \[9x^2 + 4y^2  = u^2 + v^2 = 1\]
      \[S = \left\{(u, v) | 0 \le u \le 1, 0 \le v \le 1, u^2 + v^2 \le 1\right\}\]
      Find the Jacobian.
      \[\left|\frac{\partial(x, y)}{\partial(u, v)}\right|
      = \begin{vmatrix}
        \pderiv{x}{u} && \pderiv{x}{v} \\
        \pderiv{y}{u} && \pderiv{y}{v}
      \end{vmatrix}
      = \begin{vmatrix}
        \frac{1}{3} && 0 \\
        0 && \frac{1}{2}
      \end{vmatrix}\]
      \[\iint\limits_R \sin(9x^2 + 4y^2)\ dA
      = \iint\limits_S \sin(u^2 + v^2) \left|\begin{matrix}
        \frac{1}{3} && 0 \\
        0 && \frac{1}{2}
      \end{matrix}\right|\ dA\]
      \[= \frac{1}{6}\iint\limits_S \sin(u^2 + v^2)\ dA\]
      Because $S$ is a circular region,
      we can convert the double integral into a double polar integral.
      \[r^2 = u^2 + v^2\]
      \[S = \left\{(r, \theta) | 0 \le r \le 1, 0 \le \theta \le \frac{\pi}{2}\right\}\]
      \[\frac{1}{6}\iint\limits_S \sin(u^2 + v^2)\ dA
      = \frac{1}{6}\iint\limits_S r \sin(r^2)\ dA
      = \frac{1}{6}\int_{0}^{\frac{\pi}{2}} \int_{0}^{1} r \sin(r^2)\ dr\ d\theta\]
      \[ = \frac{1}{6}
      \left[\int_{0}^{\frac{\pi}{2}}\ d\theta\right]
      \left[\int_{0}^{1} r \sin(r^2)\ dr\right]\]
      Use $u$ substitution here.
      \[u = r^2\]
      \[du = 2r\ dr\]
      \[dr = \frac{du}{2r}\]
      \[\int_0^1 r \sin(r^2)\ dr = \int_{0}^{1} \frac{1}{2} \sin(u)\ du\]
      \[ = \left[- \frac{1}{2} \cos(r^2)\right]_{r = 0}^1\]
      \[\frac{1}{6}
      \left[\int_{0}^{\frac{\pi}{2}}\ d\theta\right]
      \left[\int_{0}^{1} r \sin(r^2)\ dr\right]
      = \frac{1}{12}\pi \left[- \frac{1}{2} \cos(r^2)\right]_{r = 0}^1\]
      \[ = - \frac{1}{24}\pi\left(\cos 1 - \cos 0\right)\]
      \[ =  \frac{1}{24}\pi\left(\cos 0 - \cos 1\right)\]
      \[ =  \frac{1}{24}\pi\left(1 - \cos 1\right)\]
\end{enumerate}
\end{document}
